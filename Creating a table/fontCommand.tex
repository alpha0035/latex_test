\newcommand{\header}[1]{\textbf{#1}}

\begin{tabbing}
    Family \= \verb|\textrm{...}| \=\header{Declaration} \= \kill
    % \kill to hide that line
    \header{Type}\> \header{Command} \> \header{Declaration} \> \header{Example}\\
    Family \>\verb|textrm{...}|\>\verb|\rmfamily|\>\rmfamily Example rmfamily \\
    \>\verb|\textsf{...}|\>\verb|\sffamily|\>\sffamily Example sffamily\\
    \>\verb|\texttt{...}|\>\verb|\ttfamily|\>\ttfamily Example ttfamily
\end{tabbing}

% \verb|text| typesets code "as it is", without interpreting commands within
% \verb cannot be use in args of commands include \section and \footnote
% verbatim text use verbatim enviroment
\begin{verbatim}
    \textrm{\bfseries article}
    \begin{itemize}
        \item level 1
    \end{itemize}
\end{verbatim}

\begin{tabbing}
    
    Nametest \= test text \= level test \= level tested \= tested level\\
    \> level 1 \> level 2 \> level 3 \> level 4 \\
    \> level 1 \> level 2 \> level 3 \> level 4 \+\\
    \> level 1 \> level 2 \> level 3 \- \\%\> level 4 \\
    \> level 1 \> level 2 \> level 3 \> level 4 \+\+\\
    \> level 1 \> level 2 \\%\> level 3 \> level 4
    \< level 1 \> level 2 \> level 3 \> level 4 \\
    % \+ cause subsequent line to start at the first tab
    % Use \+\+ to start at the second tab and so on
    % \- cancel preceding \+
    % \< at the beginning of line cancel the effect of previous \+ command
\end{tabbing}
