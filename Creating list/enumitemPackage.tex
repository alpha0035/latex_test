\usepackage[shortlabels, inline]{enumitem}
% \setlist sets properties valid for all type of list.
% nolistsep to achieve very compact list analogous to compact paralist environment
\setlist{nolistsep}
% setitemize modifies all types of bulleted lists
\setitemize[1]{label=---}
% \setenumerate sets properties valid for numbered lists.
% We use it to specify label and font
% \Alph* stand for enumeration in capital letters A, B, C, ...
\setenumerate[1]{label=\textcircled{\scriptsize\Alph*}, font=\sffamily}
\begin{document}
\begin{enumerate}
    \item level 1
    \item level 1
    \begin{enumerate}[label=\Roman*., start=3]
        \item level 2
        \item level 2
    \end{enumerate}
    \begin{enumerate}[nolistsep, label=\alph*)]
        \item level 2
        \item level 2
    \end{enumerate}
    \begin{enumerate}[label=\alph*)]
        \item level 2
        \item level 2
    \end{enumerate}
    % Let's counter continue from the previous list value.
    % starring if want to folow previous formatting
    \begin{enumerate}[resume*]
        \item level 2
        \item level 2
    \end{enumerate}
    % Allows customization quickly and easily with option shortlabels
    \begin{enumerate}[(1)]
        \item level 2
        \item level 2
    \end{enumerate} 
    % to use basic list in line, add package opt inline
    % and environment enumerate*, itemize*, description*   
    \begin{enumerate*}
        \item level 2
        \item level 2
    \end{enumerate*}
\end{enumerate}
\begin{enumerate}
    \item level 1
    \begin{itemize}
        \item level 2 
        \item level 2
    \end{itemize}
\end{enumerate}
\end{document}